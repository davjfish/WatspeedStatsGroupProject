\section{Conclusions}

\subsection{Was the model useful?}

Within the geographic scope of this dataset, certain species were found to be ubiquitous while others less so.
For example, American plaice were detected at the majority of sites, while American lobster only turned up in certain
geographic areas.
It was interesting to note that latitude, longitude and elevation were still found to be significant predictors for each
species despite those differences.

The component of the analysis focusing on discrimination thresholds was interesting but less useful.
Specifically, I was surprised to see all the key thresholds for American plaice converge in the ROC graph (see Figure~\ref{fig:rocs}).
The reason why it was not useful, is that the goal of the analysis was not to predict presence-absence, per se, but instead
to produce a distribution map of based on the resulting probabilities of occurrence.


\subsection{What did you learn about your data set?}

Aquatic ecosystems face numerous challenges, from over-fishing to climate change to the introduction of invasive species.
Understanding and predicting future species distributions is an important component of fisheries management and conservation.
For example, in the face of climate change, many species distributions are changing: some are shifting, others are expanding
and other are diminishing.
This analysis has demonstrated that a statistical tool like logistic regression have the potential to be a powerful
way to model the distribution of aquatic species; both past, present and future.

In addition to water depth, it would be interesting to add other co-variates in the logistical regression models.
For example, I would expect that salinity, dissolved oxygen, water temperature and turbidity would also be useful in predicting
probability of occurrence.
%Bathymetric data is readily accessible through GEBCO, making the scope of this analysis suitable for a project such as this.
As alluded to above, adding a temporal component to this analysis would be interesting as well.

The distribution models of the six species demonstrated different patterns of occurrence; some being more heterogeneous than others.
Species like Plaice and Cod were generally observed to be ubiquitous in the sGSL, especially in the shallower waters.
For those species, there is a much lower probability of occurrence in the deeper waters of the St. Lawrence seaway.
The modeled distributions of Herring and Snow crab seemed to be more influence by latitude / longitude than by elevation.
Redfish and Lobster had sharp signals of presence-absence, i.e., a relatively narrow range of inflection in the sigmoid
logit(p) vs. p plots (Figure~\ref{fig:logitplots}).
The distribution of both of these species seemed heavily driven by elevation, albiet inversely from one another.
Redfish were predicted to occur primarily in deep waters and lobster were predicted to occur primarily in shallow waters.
This in fact matches reality very closes, as can be observed by the overlaid points on the heatmaps in Figure~\ref{fig:heat_maps}.


