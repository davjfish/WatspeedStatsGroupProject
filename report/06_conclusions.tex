\section{Conclusions}

\subsection{Was the model useful?}

Within the geographic scope of this dataset, certain species were found to be ubiquitous while others.
For example, American plaice were detected at the majority of sites, while American lobster only turned up in certain
geographic areas.
It was interesting to note that latitude, longitude and elevation were still found to be significant predictors for each
species despite those differences.

The component of the analysis focusing on discrimination thresholds was interesting but less useful.
Specifically, I was surprised to see all the key thresholds for American plaice converge at 1.0.

The reason why it was not useful, is that the goal of the analysis was not to predict presence-absence, per se, but instead
to produce a distribution map of based on the resulting probabilities of detection.


\subsection{What did you learn about your data set?}

Aquatic ecosystems face numerous challenges, from over-fishing to climate change to the introduction of invasive species.
Understanding and predicting future species distributions is an important component of fisheries management and conservation.
For example, in the face of climate change, many species distributions are changing: some are shifting, others are expanding
and other are diminishing.

This analysis has demonstrated that a statistical tool like logistic regression has the potential to be a powerful
way to model the distribution of aquatic species; both past, present and future.
In addition to water depth, it would be interesting to add other co-variates in the logistical regression model.
For example, I would expect that salinity, dissolved oxygen, water temperature and turbidity would also be useful in predicting
probability of occurrence.
Bathymetric data is readily accessible through GEBCO, making the scope of this analysis suitable for a project such as this.
As alluded to above, adding a temporal component to this analysis would be interesting as well.


The distribution models of the six species demonstrated different patterns of occurrence; some being more heterogeneous than others.


In the case of the former, American plaice being the best example, the models were of limited use.
For species like Redfish and Lobster, which are known to respond strongly to water depth, the models were seemed to be very effective.
In the heatmap for Lobster, it was interesting to note the areas which the model predicted the presence to be highest are
very well-established lobster fishing areas (hence why no samples were collected there).


%It is really cool to look at Plaice, where the model basically said \"always bet on it being there\"




Would be really interesting to look at the trends over time.