\section{Objectives}

Canadian oceans provide an important source of livelihood to Canadian in addition to fulfilling many important ecological roles.
The Canadian Government has a mandate to ensure the Oceans remain ecologically healthy as well as economically viable.
The Department of Fisheries and Oceans (DFO) is tasked with conducting regular stock assessments surveys to monitor the populations of fish and other aquatic
organisms, including crustaceans.
These surveys are carried out on either Canadian Coast Guard Science vessels or contracted out through privately owned fishing vessels conduct fishing
primarily using bottom trawls. In addition to fishing, oceanographic measurements, such as water salinity, temperature and dissolved oxygen, are collected.
The ultimate goal of these surveys is to estimate population indices for species of ecological and economic interest.
Many of the data from stock assessment surveys have been made available via the Open Government Portal~\cite{ogp}.

\subsection{Goal of the analysis}

The importance of having robust models for predicting the distribution of aquatic organisms cannot be overstated.
From a resource extraction point of view, modelled distributions can help direct fishing efforts; resulting in more efficient fisheries.
From an environmental conservation point of view, distributions modelling can help to identify which geographic areas should be targets for
conservation efforts.
The goal of this analysis is to compute models that can be used to predict the probability of occurrences of species of interest in the Southern
Gulf of St. Lawrence (sGSL).
Specifically, I employ a logistic regression approach, using latitude, longitude and water depth (i.e., elevation) to predict the probability of
occurrence of the following six species:

\begin{itemize}
    \item American plaice (\textit{Hippoglossoides platessoides})
    \item Atlantic cod (\textit{Gadus morhua})
    \item Atlantic herring (\textit{Clupea harengus})
    \item Redfish unidentified (\textit{Sebastes sp.})
    \item American lobster (\textit{Homarus americanus})
    \item Snow crab (\textit{Chionoecetes opilio})
\end{itemize}

\subsection{Rationale behind the analysis}

Spatial autocorrelation, i.e., the phenomenon when observations in space that are closer together are more correlated, is a common occurrence
in the natural world~\cite{spac_wiki}.
Accordingly, I would expect that using geographic coordinates to predict the distribution of occurrences would useful.

A basic premise in ecology is that different species can be characterized by the different ecological niches they inhabit;
each having their own strategy for making a living.
In aquatic ecosystems, the physio-chemical and ecological environments change substantially with water depth.
For example, certain species will be physiologically adapted to cope with the colder temperature, higher pressure and salinity levels that occur at
low elevations.
Accordingly, it would be reasonable to expect a species' distribution to be impacted by changes in elevation.

