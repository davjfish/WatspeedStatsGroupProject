\section{Objectives}

Canada oceans provide and important source of livelihood to Canadian as well as fulfill important ecological services.
The Canadian Government has a mandate to ensure the Oceans remain healthy and economically and ecologically viable.
The Department of Fisheries and Oceans (DFO) is tasked with conducting regular stock assessments surveys.
These survey can be carried out on either Canadian Coast Guard Science vessels or contracted through privately owned fishing vessels.
The stock assessment surveys conduct fishing and oceanographic activity as a means to monitor fish populations and produce population indices.
Many of the data from stock assessment surveys have been made available via the Open Government Portal~\cite{ogp}.


\subsection{What I am setting out to predict?}

Fish and other aquatic organisms


in all Canadian Oceans, including the Southern Gulf of St. Lawrence.



The goal of this

 What is your rationale for there being a correlation in the data that you’re looking to confirm and/or exploit?


The goal of this analysis is to explore and better understand changes in the production of agricultural crops in Canada.
Understanding these fluctuations can yield valuable insights into the Canadian economy and the agricultural sector.

Statistics Canada collects copious amounts of data via the Census of Agriculture~\cite{census} thus making these datasets excellent candidates for analysis in our group project.

\subsection{What question(s) do you want to answer?}

While the collections of elements affecting the total production of agricultural products are complex and multifaceted, this report we focus solely on a single variables: price.
Specifically, we seek to answer the question of what affect does price have on the production of different agricultural products?

\subsection{What hypothesis(es) do you have and what is your approach to tackle the problem?}

Our hypothesis is that there exists a relationship between the price of a commodity and its production in Canada, particularly for commodities without any Supply Management interference from the Government.
To tackle the problem, we sourced data from the Government's Open Data Portal, prepared the data and then created and validated a model of select products.
We then applied linear regression on that dataset to search for a relationship between price and production.
